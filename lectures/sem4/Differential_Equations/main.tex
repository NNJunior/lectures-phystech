
\hypertarget{lecture1}{}
\section{Продолжение автономных систем}

\begin{reminder}
    Мы рассматриваем автономные системы, т.е. вида \(x' = f(x)\).
\end{reminder}

\subsection{Линейные системы}
Будем рассматривать уравнение в \(\R^2\):
\begin{equation}
    x' = Ax
\end{equation}
Где \(A \in \R^{2\times 2}\) --- матрица, причем \(\det A \ne 0\) (система простая)

\subsubsection{Вещественные собственные числа у матрицы}

Для начала рассмотрим случай наличия двух собственных векторов у матрицы \(A\). Положим \(h_1, h_2\) --- собственные векторы, тогда они ЛНЗ. Заметим, что тогда:
\[x(t) = c_1e^{\lambda_1t}h_1 + c_2e^{\lambda_2t}h_2\]

Рассмотрим плоскость в базисе \((h_1, h_2)\). Пусть \(\xi_i\) --- \(i\)-ая координатная функция решения. Тогда:
\[\xi_1 = c_1e^{\lambda_1t}, \xi_2 = c_2e^{\lambda_2t} = c_2e^{\frac{\lambda_2}{\lambda_1}\lambda_1t} = c_2\left( \frac{\xi_1}{c_1} \right)^{\frac{\lambda_2}
{\lambda_1}}\]

Таким образом, в координатах \(h_1, h_2\) и при \(t \ra \infty\), мы можем видеть следующую картинку (стрелки показывают движение решения при \(t \ra \infty\)):

\paragraph{Случай \(\lambda_1 < 0, \lambda_2 < 0, |\lambda_1| < |\lambda_2|\)}

В таком случае поведение траектории в окрестности положения равновесия называется \textit{устойчиывм узлом}. Картинка в данном случае будет следующая:

\begin{center}
    тут должна быть картинка
\end{center}

\paragraph{Случай \(\lambda_1 > 0, \lambda_2 > 0, \lambda_1 < \lambda_2\)}
В таком случае поведение траектории в окрестности положения равновесия называется \textit{неустойчивым узлом}. Картинка в данном случае будет следующая:

\begin{center}
    тут должна быть картинка
\end{center}

\paragraph{Случай \(\lambda_1 < 0 < \lambda_2\)}
В таком случае поведение траектории в окрестности положения равновесия называется \textit{седлом}. Картинка в данном случае будет следующая:

\begin{center}
    тут должна быть картинка
\end{center}

\paragraph{Случай \(\lambda_1 = \lambda_2\)}
В таком случае поведение траектории в окрестности положения равновесия называется \textit{дикритическим узлом}. Картинка в данном случае будет следующая:

\begin{center}
    тут должна быть картинка
\end{center}

Для случая, когда собственный вектор один:

\paragraph{Случай одного собственного вектора с собственным числом \(\lambda\)}
Тогда пусть \(h_1\) --- собственный вектор, \(h_2\) --- присоединенный к нему. Тогда:
\[x(t) = c_1e^{\lambda t}h_1 + c_2e^{\lambda t}(h_1t + h_2) = h_1\underbrace{(c_1e^{\lambda t} + c_2e^{\lambda t}t)}_{\xi_1} + h_2 \underbrace{c_2e^{\lambda t}}_{\xi_2}\]

Тогда имеем:
\[\xi_1 = \frac{c_1}{c_2}\xi_2 + \xi_2 \frac{1}{\lambda}\ln\frac{\xi_2}{c_2}\]

В таком случае поведение траектории в окрестности положения равновесия называется \textit{дикритическим узлом}. Картинка в данном случае будет следующая:

\begin{center}
    тут должна быть картинка
\end{center}

\subsubsection{Комплекснозначные собственные числа у матрицы}
Пусть \(\lambda = \alpha + i\beta, \beta \ne 0\). Тогда \(h = h_1 \pm ih_2\) --- собственные векторы, где \(h_1, h_2\) --- ЛНЗ. Тогда:
\[x(t) = ce^{\lambda t}h + \overline{c}e^{\overline{\lambda}t}\overline{h}, c \in \Cm\]
Положим \(c = \frac{r}{2}e^{i\phi}\), тогда:
\[x(t) = \frac{r}{2}e^{\alpha t}\left( e^{i\phi + i\beta t}(h_1 + ih_2) e^{-i\phi - i\beta t}(h_1 - ih_2) \right) = re^{\alpha t}(\cos(\phi + \beta t)h_1 - \sin(\phi + \beta t)h_2)\]
Получаем:
\[\xi_1(t) = re^{\alpha t}\cos(\phi + \beta t)\]
\[\xi_2(t) = re^{\alpha t}\sin(\phi + \beta t)\]

Картинки в зависимости от ранзых \(\alpha, \beta\) будут следующие:
\begin{center}
    тут должны быть картинки
\end{center}

\subsection{Нелинейные системы}
Будем рассматривать уравнение
\begin{equation}
    x' = f(x), f: \Omega \ra \R^2, f \in C^2, \Omega\text{ --- открытое}
\end{equation}
Пусть \(\tilde{x} \in \Omega\) таково, что \(f(\tilde{x}) = 0\), т.е. \(\tilde{x}\) --- положение равновесия. Рассмотрим еще одно уравнение:
\begin{equation}
    y' = f'(\tilde{x})y, f'(x) = \left( \begin{array}{cc}
        \frac{\partial f_1}{\partial x_1}(x) & \frac{\partial f_1}{\partial x_2}(x) \\ 
        \frac{\partial f_2}{\partial x_1}(x) & \frac{\partial f_2}{\partial x_2}(x) \\ 
    \end{array} \right)
\end{equation}

\begin{theorem}[б/д]
    Пусть \(\det f(\tilde{x}) \ne 0\), (т.е. система (1.3) простая), причем \(0\) не является ее центром. Тогда \(\exists U\) --- окрестность \(\tilde{x}\), \(\exists V\) --- окрестность \(0\), \(\exists \psi: U \ra V\) --- гомеоморфизм, такие, что выполнены следующие условия:
    \begin{enumerate}
        \item \(\forall\) траектории \(X \subset V\) системы (1.2), \(\psi(X)\) --- тракетория системы (1.3)
        \item \(\forall\) траектории \(Y \subset U\) системы (1.3), \(\psi^{-1}(Y)\) --- тракетория системы (1.2)
    \end{enumerate}
\end{theorem}

\begin{example}
    Рассмотрим систему:
    \begin{equation*}
        \begin{cases*}
            x_1' = -x_2 - x_1|x| \\
            x_2' = x_1 - x_2|x|
        \end{cases*}, \tilde{x} = \left( \begin{array}{c}
            0 \\
            0
        \end{array} \right)
    \end{equation*}

    Тогда:
    \[f'(0) = \left( \begin{array}{cc}
        0 & -1 \\
        1 & 0
    \end{array} \right)\]
    И линеаризованная система имеет вид:
    \[y' = \left( \begin{array}{cc}
        0 & -1 \\
        1 & 0
    \end{array} \right)y\]
    Заметим, что тогда \(\lambda \pm i\) и \(0\) --- центр. Сделаем замену \(x_1 = r\cos\phi, x_1 = r\sin\phi\), имеем:
    \begin{eqnarray*}
        \begin{cases*}
            r'\cos\phi - r(\sin\phi)\phi' = -r\sin\phi - r^2\cos\phi \\
            r'\sin\phi + r(\cos\phi)\phi' = r\cos\phi - r^2\sin\phi
        \end{cases*}
    \end{eqnarray*}
    \begin{eqnarray*}
        \begin{cases*}
            r' = -r^2
            -r\phi' = -r \Ra \phi' = 1
        \end{cases*}
    \end{eqnarray*}
    Получили, что \(\phi = \phi_0 + t\) и 
    \[x(t) = \left( \begin{array}{c}
        r(t)\cos(\phi_0 + t) \\
        r(t)\sin(\phi_0 + t) \\
    \end{array} \right)\]
    Таким образом, картинка будет следующей:
    \begin{center}
        тут должна быть картинка
    \end{center}
    Т.е. одна траектория получилась не замкнутой. Но тогда между ними не может существовать гомеоморфизма. Таким образом, условие, что 0 --- не центр существенно.
\end{example}

